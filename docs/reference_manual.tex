%%%%%%%%%%%%%%%%%%%%%%%%%%%%%%%%%%%%%%%%%%%%%%%%%%%%%%%%%%%%%%%%%%%%%%%%%%%%%%%
% Non-blocking Ada Reference Manual
%
% Copyright (C) 2007 - 2008  Anders Gidenstam (andersg@mpi-inf.mpg.de)
%
%  Algorithms and Complexity Group
%  Max-Planck-Institut f�r Informatik
%  Stuhlsatzenhausweg 85
%  66123 Saarbr�cken, Germany
%
% $Id: reference_manual.tex,v 1.15 2008/09/16 15:52:02 andersg Exp $
%
%%%%%%%%%%%%%%%%%%%%%%%%%%%%%%%%%%%%%%%%%%%%%%%%%%%%%%%%%%%%%%%%%%%%%%%%%%%

\documentclass[english,10pt]{book}

%%%%%%%%%%%%%%%%%%%%%%%%%%%%%%%%%%%%%%%%%%%%%%%%%%%%%%%%%%%%%%%%%%%%%%%%%%%

\usepackage[english]{babel}
\usepackage{amssymb}
\usepackage{amsmath}
%\usepackage{graphicx}   % For importing graphics
\usepackage{moreverb}   % To be able to use the environment boxedverbatim
\usepackage{subfigure}  % To be able to use \subfigure
\usepackage{fancyhdr}
\usepackage{makeidx}    % To make an index
\usepackage{float}
\usepackage{a4wide}
\usepackage{listings}
\lstloadlanguages{Ada}
%\usepackage[chapter]{algorithm}
%\usepackage[noend]{algorithmic}
%\usepackage{color}
%\usepackage{psfig}
%\usepackage{psfrag}
%\usepackage{epsfig}

%%%%% Typesets operation and variable names in the main text.
\newcommand{\code}[1]{{\textsf{{#1}}}}
\newcommand{\cmd}[1]{{\tt {#1}}}

%%%%% Typesets keywords in code figures.
\newcommand{\kw}[1]{{\textbf{#1}}}
\newcommand{\term}[1]{{\emph{#1}}}

%%%%% for easy-to-undo modifications
\newcommand{\remove}[1]{{}}

%%%%% For eye-catching TODO notes, questions, etc
\newcommand{\todo}[1]{\noindent{\bf TODO:}{\sc #1}}
%\newcommand{\todo}[1]{{}}

%%%%%%%%%%%%%%%%%%%%%%%%%%%%%%%%%%%%%%%%%%%%%%%%%%%%%%%%%%%%%%%%%%%%%%%%%%%
%%%%% Document specific definitions

%% Library name
\newcommand{\nbada}{{\sc NbAda}}

%% Ada
\newcommand{\ada}{{\sc Ada}}

%% Program float environment
\floatstyle{ruled}
\newfloat{AdaProgram}{thp}{lop}
\floatname{AdaProgram}{Program}

%% Ada code environment
\newenvironment{AdaCode}
               {\begin{small}}
               {\end{small}}

%% Text output environment
\newenvironment{TextOutput}
               {\begin{small}}
               {\end{small}}

%%%%% For the draft version.
\newif\ifreleaseA
\releaseAtrue


%%%%% Make index
%\makeindex

%%%%%%%%%%%%%%%%%%%%%%%% EOD %%%%%%%%%%%%%%%%%%%%%%%%%%%%%%%%%%%%%%%%%%%%%%

\begin{document}

\selectlanguage{english}

%%%%%%%%%%%%%%%%%%%%%%%%%%%%%%%%%%%%%%%%%%%%%%%%%%%%%%%%%%%%%%%%%%%%%%%%%%%
%% Title page

\begin{titlepage}
  
  \title{%
    \Huge \nbada\\
    \vspace{5mm}
    \huge Non-blocking Algorithms and Data Structures Library\\
    \vspace{5mm}
    \huge Reference Manual\\
    \ifreleaseA
    \Large version 0.1.0-pre0
    \vspace{5mm}
    \fi
  }
  \author{%
    Anders Gidenstam  (andersg(at)mpi-inf.mpg.de)
  }

  \date{
    Draft \today\\
    \vspace{5mm}
    %%
    \vspace{12mm}
    Algorithms and Complexity Group\\
    Max-Planck-Institut f\"ur Informatik\\
    Stuhlsatzenhausweg 85\\
    66123 Saarbr\"ucken, Germany\\
    \vspace{12mm}
    Saarbr\"ucken, Germany, 2008
  }


  \maketitle
\end{titlepage}

\lstset{language=Ada,
        defaultdialect=[95]Ada,
        basicstyle=\footnotesize,
        numbers=left,
        numberstyle=\tiny,
        stepnumber=2,
        numbersep=5pt}

%%%%%%%%%%%%%%%%%%%%%%%%%%%%%%%%%%%%%%%%%%%%%%%%%%%%%%%%%%%%%%%%%%%%%%%%%%%
\chapter{Introduction}\label{chpt:Introduction}

\nbada\ is a collection of non-blocking algorithms and concurrent
data structures
in a common infrastructure aiming to be accessible for use by
non-expert programmers as well as providing support for implementation
of further non-blocking algorithms. \nbada\ is implemented in \ada\ 95 and
is distributed as free software under the terms of the GNU Public License
(c.f. Appendix~\ref{chpt:License}).

%%%%%%%%%%%%%%%%%%%%%%%%%%%%%%%%%%%%%%%%%%%%%%%%%%%%%%%%%%%%%%%%%
%% From my thesis. Reduce the blabla.
\section{Concurrent data access in shared memory systems}%
\label{background:sec:shared-memory_coordination}%
\footnote{This introduction is based on the introduction
  in~\cite{Gidenstam:2006:PhD}}

In a shared memory system the processes\footnote{We will use the term
  process and thread interchangeably in the context of general shared
  memory synchronization. If we talk about threads and processes in the
  operating system sense it will be made clear from context.}
have access to a set of shared memory locations which they may use to
communicate.  A process can \emph{read} data from and
\emph{write} data to each shared memory location.  The number of
processes can be much larger than the number processors due to
multiprogramming, which may interleave the execution of several
processes on the same processor. The processes are often considered to
be asynchronous, that is, their rate of execution might vary
arbitrarily, because of the interleaving. This has certain
implications for the possibilities for the synchronization and
coordination of processes which we will discuss below.

%%%%%%%%%%%%%%%%%%%%%%%%%%%%%%%%%%%%%%%%%%%%%%%%%%%%%%%%%%%%
\subsection{Linearizability}

We want the semantics of all operations on a shared data
object to be the same as for the same operation on the 
corresponding sequential object.
%
The consistency model that captures this
is called \emph{linearizability} and was introduced by Herlihy and Wing
in~\cite{HerlihyWing:1990:LCC}. Linearizability requires that
for each operation, in a concurrent execution of operations on the
shared data object, there is an atomic time instant that lies within
its duration where the operation takes effect, in a way such that the
outcome of the operation agrees with the object's sequential
specification.


%%%%%%%%%%%%%%%%%%%%%%%%%%%%%%%%%%%%%%%%%%%%%%%%%%%%%%%%%%%%
\subsection{Lock-based synchronization}%
\label{background:subsec:lock-based}

The traditional way to synchronize processes/threads accessing a
shared data object in a concurrent program is
to use \emph{mutual exclusion}. Mutual exclusion is normally implemented
using a lock, which is a
shared variable together with routines to atomically \emph{acquire} and
\emph{release} the lock. The atomicity of \emph{acquire} and \emph{release}
guarantees that only one process can acquire and hold the lock at a time.
The most common approach when synchronizing using locks is
to use the lock to implement \emph{critical sections}, that is, some pieces
of code that can only be run by one process at the time.
For a shared data object, it is common that the operations it supports
are implemented as mutually exclusive critical sections.

The use of locks and the sequential nature of critical sections cause
a number of drawbacks, namely:
\begin{itemize}
\item {\bf Deadlock prone.} With locks it is not hard to create circular lock
  dependencies that cause two (or more processes)
  to get blocked by both trying to acquire a lock that is held by the other.
  Furthermore, a process that crashes while holding some lock(s) is
  also likely to block the progress of other processes.
  
\item {\bf Blocking.} The process that has acquired the lock will delay
  all other processes that also need that lock until it has finished
  executing inside the critical section.  To make matters worse the process
  inside the critical section may itself be delayed by being preempted by the
  scheduler, suffer a page-fault, try to acquire another lock or wait for IO
  inside the critical section. 
  
\item {\bf Priority inversion.} This is a pathological case that can occur
  when using a strict priority based scheduler, where a
  medium priority process can delay a high priority process, potentially
  indefinitely on a single processor system, by preempting a low priority
  process that has acquired a lock needed by the high priority process.
  This problem can be avoided by employing
  \emph{priority inheritance protocols} as proposed by
  Sha et al.~\cite{ShRaLe:1990:PIP}.
\end{itemize}


%%%%%%%%%%%%%%%%%%%%%%%%%%%%%%%%%%%%%%%%%%%%%%%%%%%%%%%%%%%%
\subsection{Non-blocking synchronization}

Non-blocking synchronization techniques avoid the use of locks by
using cunning algorithms, which often but not always use hardware
synchronization primitives, to create shared data objects that can be
accessed simultaneously by several processes.  By avoiding locks
non-blocking synchronization does not exhibit the problems of
deadlocks, blocking and priority inversion, which lock-based
synchronization suffers from. Non-blocking shared data objects also have a
higher degree of fault-tolerance than lock-based ones since they can tolerate
any number of processes experiencing stop-failures. 

There are two kinds of non-blocking synchronization,
\emph{lock-free} synchronization and the stronger \emph{wait-free}
synchronization. 

%%%%%%%%%%%%%%%%%%%%%%%%%%%%%%%%%%%%%%%%%%%%%%%%%%%%%%%
\subsubsection{Lock-free synchronization}

A \emph{lock-free} algorithm guarantees that regardless of the contention
caused by concurrent operations and the interleaving of their steps, at
each point in time there is at least one operation which is able to make
progress. However, as there is no fairness guarantee, some
operation could be starved and take unbounded time to finish.

The lack of fairness guarantee significantly simplifies the construction of
lock-free algorithms compared to wait-free ones and leads to algorithms that
are fast
when there are no conflicts but cause slow down for all except one process
involved in a conflict. Hence, lock-free synchronization is also known as
\emph{optimistic synchronization}~\cite{Rinard99}.

In~\cite{Herlihy93} Herlihy described a general method for
transforming any sequential data object implementation to a lock-free shared
data object implementation.
In short, the methodology is like this: The state of the shared data object is
represented by a pointer to the current version; an operation on the shared
data object first makes a new private copy of the current version, applies
the sequential version of the desired operation on the private copy and thus
creates a new prospective state of the shared object. 
Then it tries to install this prospective state as the new version of the
shared object using an atomic synchronization primitive that will only
succeed if the current version of the shared object is still the same as the
one the new state was computed from. If the operation fails to install its
new state, some other operation(s) have managed to install their new versions
and this operation has to retry from the beginning.

This general methodology is often not very efficient because (i) the
entire object is copied for each update (this can be optimized though)
and (ii) the resulting lock-free shared object is not
\emph{disjoint-access parallel}, that is, all
concurrent operations on it cause conflicts even when the operations
only access disjoint parts of the shared object.

For the above reasons, a significant research effort is being spent on the
task of designing and developing efficient lock-free implementations of
various data structures.

The use of lock-free instead of lock-based synchronization can give
significant performance gains in parallel applications, as shown by
Tsigas and Zhang in~\cite{TsiZ01,TsiZ02}, as well as in operating systems,
for example as suggested by Greenwald and Cheriton in~\cite{GreenwaldChe96}.

%%%%%%%%%%%%%%%%%%%%%%%%%%%%%%%%%%%%%%%%%%%%%%%%%%%%%%%
\subsubsection{Wait-free synchronization}

A \emph{wait-free} algorithm is both lock-free and \emph{fair}, it
guarantees that every operation finishes in a bounded number of its
own steps, regardless of the actions of other operations.
%
This is a very strong property, as it decouples the processes using
the same shared data object from each other. This makes wait-free shared data
objects attractive to use, for example, in hard \emph{real-time systems}
where the worst-case execution time has to be known for every operation and
where lock-based critical sections limit the schedulability of the system and
complicate the schedulability analysis.
A drawback, however, is that algorithms that are wait-free, are 
often also quite complex, in particular for non-trivial shared objects.

A common approach in implementing wait-free algorithms is the use of
\emph{helping schemes}~\cite{Herlihy91}.  In a helping
scheme each operation first announces information about what it wants
to do with the shared object in some global data structure, then it
checks in the announce-structure to see if there are other operations
that it needs to help before proceeding with its own.

Barnes presented a method similar to helping
in~\cite{Barnes:1993:AMFI}. In his method each operation on the shared
data object is divided into a sequence of \emph{virtually atomic
suboperations}, where each suboperation is constructed so that
once it has begun, it is guaranteed to be performed fully, either by the
initiating process or by being helped by another process.

In~\cite{Herlihy91} Herlihy presented a universal method for constructing
a wait-free algorithm for any shared data object. However, as for the general
methodology for construction of lock-free algorithms, the universal
construction for wait-free algorithms is not practical in all cases and
therefore significant research efforts are being spent on
developing efficient wait-free algorithms. 


%%%%%%%%%%%%%%%%%%%%%%%%%%%%%%%%%%%%%%%%%%%%%%%%%%%%%%%%%%%%%%%%%%%%%%%%%%%
\chapter{Installation}\label{chpt:Install}

Installation of \nbada\ is simple: just extract the distribution
archive anywhere you want. To make it convenient to compile programs
which use components from \nbada\ there is a utility, \cmd{nbada\_config},
that outputs suitable command-line options for use with \cmd{gnatmake}.

The utility \cmd{nbada\_config}, located in \cmd{src/util}, is written
in \ada, so it needs to be compiled and the binary installed somewhere
convenient (e.g. /usr/local/bin). Before compiling the path to the
directory where the \nbada\ source code is install needs to be entered
into \cmd{nbada\_config.adb} by modifying the following line in
\cmd{nbada\_config.adb}:
\begin{AdaCode}
\begin{lstlisting}[stepnumber=0]{}
   --  NBAda source code base directory.
   Install_Base : constant String :=
     "/usr/local/share/NBAda/src";  -- Change this line.
   --  Default architecture.
   Default_Architecture : constant Architecture := IA32;
\end{lstlisting}
\end{AdaCode}
For convenience the default architecture can also be changed there, the
full list of supported architectures is shown in Table~\ref{table:ISAs}.
The \cmd{nbada\_config.adb} can be compiled, e.g. with the command
`\cmd{gnatmake nbada\_config.adb}'.

%%%%%%%%%%%%%%%%%%%%%%%%%%%%%%%%%%%%%%%%%%%%%%%%%%%%%%%%%%%%%%%%%%%%%%
\section{Using \cmd{nbada\_config}}

%%%%%%%%%%%%%%%%%%%%%%%%%%%%%%%%%%%%%%%%%%%%%%%%%%%%%%%%%%%%
\begin{table}[t]
  \center
  \begin{small}
    \begin{tabular}{|l|l|}
      \hline
      Architecture      & Description\\
      \hline
      \cmd{IA32}        & 32-bit Intel x86 Architecture
                          (Intel Pentium and above).\\
      \cmd{SPARCV8PLUS} & 32-bit mode on SPARC v9 compatible processor.\\
      \cmd{SPARCV9}     & 64-bit mode on SPARC v9 compatible processor.\\
      \cmd{MIPSN32}     & 32-bit mode on 64-bit MIPS processor
                          (e.g. R-$10000$)\\
      \hline
    \end{tabular}
  \end{small}
  \caption{Supported instruction set architectures.}
  \label{table:ISAs}
\end{table}
%%%%%%%%%%%%%%%%%%%%%%%%%%%%%%%%%%%%%%%%%%%%%%%%%%%%%%%%%%%%
\begin{table}[t]
  \center
  \begin{small}
    \begin{tabular}{|l|l|c|}
      \hline
      Option                     & Description & Chapter \\
      \hline
      \cmd{PRIMITIVES}           & Hardware atomic primitives.
      & \ref{chpt:Primitives}\\
      \cmd{LF\_POOLS}            & Lock-free storage pools.
      & \ref{chpt:Memory-allocation}\\
      \cmd{EBMR}                 & Epoch-based memory reclamation
                                   \cite{Fraser:2004:PLF}.
      & \ref{chpt:Memory-reclamation}\\
      \cmd{HPMR}                 & Hazard pointers memory reclamation
                                   \cite{Michael:2004:SMR,Michael:2002:SMR}.
      & \ref{chpt:Memory-reclamation}\\
      \cmd{PTB}                  & Pass the buck memory reclamation
                                   \cite{Herlihy:2002:ROP,HerlihyLuMaMo:2005:NBMM}.
      & \ref{chpt:Memory-reclamation}\\
      \cmd{LFRC}                 & Lock-free reference counting memory
                                   reclamation
                                   \cite{HerlihyLuMaMo:2005:NBMM}.
      & \ref{chpt:Memory-reclamation}\\
      \cmd{LFMR}                 & Lock-free reference counting memory
                                   reclamation
                                   \cite{GiPaSuTs:2005:LFGC,GiPaSuTs:2008:LFMR}.
      & \ref{chpt:Memory-reclamation}\\
      \cmd{SW\_LL\_SC}           & Lock-free load-linked/store-conditional
                                   primitive
                                   \cite{Michael:2004:PLFLL}.
      & \ref{chpt:LFLLSC}\\
      \cmd{LF\_STACKS\_EBMR}     & Lock-free dynamic stack
                                   \cite{IBM:1983,Michael:2004:SMR}.
      & \ref{chpt:LFSTACKS}\\
      \cmd{LF\_STACKS\_HPMR}     & Lock-free dynamic stack
                                   \cite{IBM:1983,Michael:2004:SMR}.
      & \ref{chpt:LFSTACKS}\\
      \cmd{LF\_QUEUES\_BOUNDED}  & Lock-free bounded queue
                                   \cite{TsigasZhang2001:SPAA}.
      & \ref{chpt:LFQUEUES}\\
      \cmd{LF\_QUEUES\_EBMR}     & Lock-free dynamic queue
                                   \cite{Michael:1996:PODC}.
      & \ref{chpt:LFQUEUES}\\
      \cmd{LF\_QUEUES\_HPMR}     & Lock-free dynamic queue
                                   \cite{Michael:1996:PODC}.
      & \ref{chpt:LFQUEUES}\\
      \cmd{LF\_QUEUES\_LFRC}     & Lock-free dynamic queue
                                   \cite{HoffmanShSh:2007:TBQ}.
      & \ref{chpt:LFQUEUES}\\
      \cmd{LF\_QUEUES\_LFMR}     & Lock-free dynamic queue
                                   \cite{HoffmanShSh:2007:TBQ}.
      & \ref{chpt:LFQUEUES}\\
      \cmd{LF\_DEQUES\_LFRC}     & Lock-free dynamic deque
                                   (a.k.a double ended queue) \cite{SunT04c}.
      & \ref{chpt:LFDEQUES}\\
      \cmd{LF\_DEQUES\_LFMR}     & Lock-free dynamic deque
                                   (a.k.a double ended queue) \cite{SunT04c}.
      & \ref{chpt:LFDEQUES}\\
      \cmd{LF\_PRIORITY\_QUEUES\_EBMR} & Lock-free dynamic priority queue.
      & \ref{chpt:LFPRIORITYQUEUES}\\
      \cmd{LF\_PRIORITY\_QUEUES\_HPMR} & Lock-free dynamic priority queue.
      & \ref{chpt:LFPRIORITYQUEUES}\\
      \cmd{LF\_SETS\_EBMR}       & Lock-free dynamic set
                                   \cite{Michael:2002:HPDLF}.
      & \ref{chpt:LFSETS}\\
      \cmd{LF\_SETS\_HPMR}       & Lock-free dynamic set
                                   \cite{Michael:2002:HPDLF}.
      & \ref{chpt:LFSETS}\\
      \cmd{LF\_DICTIONARIES\_EBMR} & Lock-free dynamic dictionary
                                           \cite{Michael:2002:HPDLF}.
      & \ref{chpt:LFMAPS}\\
      \cmd{LF\_DICTIONARIES\_HPMR} & Lock-free dynamic dictionary
                                           \cite{Michael:2002:HPDLF}.
      & \ref{chpt:LFMAPS}\\
      \hline
    \end{tabular}
  \end{small}
  \caption{Include library options for \cmd{nbada\_config}.}
  \label{table:modules}
\end{table}
%%%%%%%%%%%%%%%%%%%%%%%%%%%%%%%%%%%%%%%%%%%%%%%%%%%%%%%%%%%%

As mentioned above, \cmd{nbada\_config} outputs command line options
for use with \cmd{gnatmake}. A typical usage pattern would be:

\begin{TextOutput}
\begin{verbatim}
% gnatmake myprogram.adb `nbada_config LF_SETS`
\end{verbatim}
\end{TextOutput}

where the shell replaces \cmd{`nbada\_config LF\_SETS`} with the output of
\cmd{nbada\_config LF\_SETS}.

The full set of options recognized by \cmd{nbada\_config} is outlined below and
explained in Table~\ref{table:ISAs} and Table~\ref{table:modules}.
\begin{TextOutput}
\begin{verbatim}
Usage: nbada_config [OPTIONS] [LIBRARIES]
Options:
        [--isa=<IA32|SPARCV8PLUS|SPARCV9|MIPSN32>]
        [--help]
Libraries:
        PRIMITIVES    (default)
        LF_POOLS
        EBMR
        HPMR
        PTB
        LFRC
        LFMR
        SW_LL_SC
        LF_STACKS_EBMR
        LF_STACKS_HPMR
        LF_QUEUES_BOUNDED
        LF_QUEUES_EBMR
        LF_QUEUES_HPMR
        LF_QUEUES_LFMR
        LF_QUEUES_LFRC
        LF_DEQUES_LFMR
        LF_DEQUES_LFRC
        LF_PRIORITY_QUEUES_EBMR
        LF_PRIORITY_QUEUES_HPMR
        LF_SETS_EBMR
        LF_SETS_HPMR
        LF_DICTIONARIES_EBMR
        LF_DICTIONARIES_HPMR
\end{verbatim}
\end{TextOutput}



\subsection*{Examples}

To compile the \nbada\ \cmd{queue\_test} micro-benchmark
(\cmd{src/benchmarks/Queues}) using a lock-free queue algorithm with
epoch-based memory reclamation the following command line could be used:

\begin{TextOutput}
\begin{verbatim}
% gnatmake queue_test -ILock-Free_Queue `nbada_config LF_QUEUES_EBMR`
\end{verbatim}
\end{TextOutput}

The argument \cmd{-ILock-Free\_Queue} is used to select which queue
implementation the benchmark will use as it can be compiled with several
different ones. Here is the command line to build with the same lock-free queue
algorithm but with the hazard pointers memory reclamation scheme:

\begin{TextOutput}
\begin{verbatim}
% gnatmake queue_test -ILock-Free_Queue `nbada_config LF_QUEUES_HPMR`
\end{verbatim}
\end{TextOutput}

This command line builds the \cmd{queue\_test} benchmark with a
bounded lock-free queue algorithm:

\begin{TextOutput}
\begin{verbatim}
% gnatmake queue_test -ILock-Free_Bounded_Queue `nbada_config LF_QUEUES_BOUNDED LF_POOLS`
\end{verbatim}
\end{TextOutput}



%%%%%%%%%%%%%%%%%%%%%%%%%%%%%%%%%%%%%%%%%%%%%%%%%%%%%%%%%%%%%%%%%%%%%%%%%%%
\chapter{Data structures}\label{chpt:Datastructures}



%%%%%%%%%%%%%%%%%%%%%%%%%%%%%%%%%%%%%%%%%%%%%%%%%%%%%%%%%%%%%%%%%%%%%%
\section{Atomic Objects}

%%%%%%%%%%%%%%%%%%%%%%%%%%%%%%%%%%%%%%%%%%%%%%%%%%%%%%%%%%%%%%%%%
\subsection{Large Register}\label{chpt:MWR}

An atomic register is a multi-word object that can be read and written
with non-blocking atomic operations.


\remove{ %% This is from my thesis. It needs to be filtered.
A fundamental problem for concurrent shared memory systems that has
received a significant amount of research effort is the
\emph{readers/writers problem}.
%%
In this problem a number of concurrent processes are interested in
reading from and/or writing to a shared data object also called a
\emph{register}. All read or write operations should take effect
atomically and they return or update the entire state of the shared
data object. For small shared data objects that fit in a single memory
word (of the word size supported by the multiprocessor system at hand)
the hardware read/write instructions and, if needed, the
memory barrier instructions described in
Section~\ref{background:sec:memory_consistency} above provide the properties
required. If, on the other hand, the shared data object is larger than
a single word (of the word size supported by the multiprocessor system
at hand) a software algorithm is needed to solve the readers/writers
problem.

The classical solution is to use mutual exclusion to enforce that
either (i)~no read or write operations overlap each other; or (ii)~no
write operations overlap each other or any read operation. These
methods, normally implemented using a \emph{mutual exclusion lock} or
a \emph{readers-writers lock}, respectively, both suffer from the
drawbacks of mutual exclusion which are further discussed in
Section~\ref{background:subsec:lock-based}.

In~\cite{Lam77} Lamport introduced a solution to the readers/writers
problem with one writer which did not use mutual exclusion.  Lamport's
algorithm allows the writer unimpeded access to the register
regardless of what the readers do, while the readers will never
interfere which each other but can be forced to retry if the writer
interferes with them.  This can force a slow reader to retry
indefinitely.  Lamport's algorithm marked the start of long running
research efforts to construct solutions to the readers/writers problem
where neither readers nor writer could be delayed indefinitely by
interference from other readers or writers.

This has made this problem, also known as the problem of multi-word wait-free
read/write registers, one of the well-studied problems in the
area of non-blocking synchronization, with numerous results for the
construction of e.g.:\\
%%
(i)~single-writer single-reader registers
\cite{Lam86,Simpson90,Chen97};\\
(ii)~single-writer $n$-reader registers
\cite{Pet83,BurP87,KirKV87,New87,Kopetz93,SinAG94,HalV95,LaGiHaPaTs:ESA04};\\
(iii)~$2$-writer $n$-reader registers~\cite{Blo88}; and\\
(iv)~$m$-writer $n$-reader registers
\cite{VitA86,PetB87,IsrS92,LiV92,LiTV96,HalV96}.

The main goal of most of the algorithms in these results is to construct
wait-free multi-word read/write registers from single-word read/write
registers and not from other synchronization primitives which may be
provided by the hardware in a system.  This has been very significant,
providing fundamental results in the area of wait-free
synchronization, in particular considering the nowadays well-known
and well-studied hierarchy of shared data objects and their
synchronization power~\cite{Herlihy91}. Many of these solutions also
involve elegant and symmetric ideas and have formed the basis for
further results in the area of non-blocking synchronization.
}%% end-remove

%%%%%%%%%%%%%%%%%%%%%%%%%%%%%%%%%%%%%%%%%%%%%%%%%%%%%%%%%%%%
\subsubsection*{The package NBAda.Atomic\_Single\_Writer\_Registers}

\nbada\ provides two implementations of linearizable single writer multiple
reader multi-word registers:
\begin{itemize}
\item Peterson's register algorithm~\cite{Pet83}; and
\item the ReaderField algorithm by Larsson et al.~\cite{LaGiHaPaTs:ESA04}.
\end{itemize}

The atomic register implementations in \nbada\ have the same public
package specification, so an application can be compiled against any
of them without source code changes.

\begin{AdaCode}
\begin{lstlisting}
generic
   type Element_Type is private;
package NBAda.Atomic_Single_Writer_Registers is

   type Atomic_1_M_Register (No_Of_Readers : Positive) is limited private;

   type Reader_Id is private;

   procedure Write (Register : in out Atomic_1_M_Register;
                    Value    : in     Element_Type);
   procedure Read  (Register : in out Atomic_1_M_Register;
                    Reader   : in     Reader_Id;
                    Value    :    out Element_Type);

   function  Register_Reader (Register : in Atomic_1_M_Register)
                             return Reader_Id;
   procedure Deregister_Reader (Register : in out Atomic_1_M_Register;
                                Reader   : in     Reader_Id);

   Maximum_Number_Of_Readers_Exceeded : exception;

private

   ...  --  Implementation details.

end NBAda.Atomic_Single_Writer_Registers;
\end{lstlisting}
\end{AdaCode}

\paragraph{Application constraints:}
\begin{itemize}
\item Concurrent calls to \code{Write} on the same atomic register are
  forbidden.
\item Concurrent calls to \code{Read} on the same atomic register 
  with the same \code{Reader\_Id} argument are forbidden.
\item \code{Reader\_Id} values should not be passed between tasks.
\item \code{Register\_Reader}/\code{Deregister\_Reader} should be used as
  seldom as possible.
\end{itemize}

%%%%%%%%%%%%%%%%%%%%%%%%%%%%%%%%%%%%%%%%%%%%%%%%%%%%%%%%%%%%%%%%%
\subsection{Linearizable Snapshots}\label{chpt:MWSS}

A snapshot is a composite data structure consisting of a number
of fields. Each field can be written separately and the entire state
of the composite can be read atomically.

\remove{
\begin{itemize}
\item 
\end{itemize}
} %% end-remove

%%%%%%%%%%%%%%%%%%%%%%%%%%%%%%%%%%%%%%%%%%%%%%%%%%%%%%%%%%%%
\subsubsection*{The package NBAda.Atomic\_Multiwriter\_Snapshots}

The \nbada\ package \code{NBAda.Atomic\_Multiwriter\_Snapshots}
implements the multiple writer per
component multiple scanner lock-free linearizable snapshot
algorithm by Jayanti~\cite{Jayanti:2005:MWS}. 


\begin{AdaCode}
\begin{lstlisting}
generic
   Max_Number_Of_Components : Natural;
   --  Maximum number of components in the snapshot.
   with package Process_Ids is
     new Process_Identification (<>);
   --  Process identification.
package NBAda.Atomic_Multiwriter_Snapshots is

   type Snapshot (<>) is private;

   function Scan return Snapshot;

   Maximum_Number_Of_Components_Exceeded : exception;

   generic
      --  Use pragma Atomic and pragma Volatile for Element.
      --  Element'Object_Size MUST be System.Word_Size.
      type Element is private;
   package Element_Components is

      type Element_Component is private;

      function Create (Default_Value : in Element) return Element_Component;

      procedure Write (To    : in Element_Component;
                       Value : in Element);

      function Read (Component : in Element_Component;
                     From      : in Snapshot) return Element;

   private

      ...  --  Implementation details.

   end Element_Components;

private

   ...  --  Implementation details.

end NBAda.Atomic_Multiwriter_Snapshots;
\end{lstlisting}
\end{AdaCode}

\paragraph{Application constraints:}
\begin{itemize}
\item Any task that calls an operation in
  \code{NBAda.Atomic\_Multiwriter\_Snapshots} must have registered an
  identity by calling the operation \code{Register} of the
  appropriate instance of \code{NBAda.Process\_Identification}.
\item All types used for components must have an \code{Object\_Size}
  equal to \code{System.Word\_Size}.
\end{itemize}

%%%%%%%%%%%%%%%%%%%%%%%%%%%%%%%%%%%%%%%%%%%%%%%%%%%%%%%%%%%%%%%%%
\subsection{Software Load-Linked/Store-Conditional for multi-word Objects}%
\label{chpt:LFLLSC}


%%%%%%%%%%%%%%%%%%%%%%%%%%%%%%%%%%%%%%%%%%%%%%%%%%%%%%%%%%%%
\subsubsection*{The package NBAda.Large\_Primitives}

The package \code{NBAda.Large\_Primitives} implements the lock-free
load-linked store-conditional algorithm by Michael~\cite{Michael:2004:PLFLL}.

The algorithm relies on lock-free memory reclamation and the
implementation uses the \code{NBAda.Hazard\_Pointers} memory
reclamation algorithm. The include flag for \cmd{nbada\_config} is
\cmd{SW\_LL\_SC}.

\begin{AdaCode}
\begin{lstlisting}
generic
   Max_Number_Of_Links : Natural;
   --  Maximum number of simultaneous LL/SC per thread.
   with package Process_Ids is
     new Process_Identification (<>);
   --  Process identification.
package NBAda.Large_Primitives is

   package MR is < Implementation defined >

   generic
      type Element is private;
   package Load_Linked_Store_Conditional is

      type Shared_Element is limited private;

      function  Load_Linked (Target : in Shared_Element) return Element;
      function  Load_Linked (Target : access Shared_Element) return Element;

      function  Store_Conditional (Target : in     Shared_Element;
                                   Value  : in     Element) return Boolean;
      function  Store_Conditional (Target : access Shared_Element;
                                   Value  : in     Element) return Boolean;

      procedure Store_Conditional (Target : in out Shared_Element;
                                   Value  : in     Element);
      procedure Store_Conditional (Target : access Shared_Element;
                                   Value  : in     Element);


      function  Verify_Link (Target : in Shared_Element) return Boolean;
      function  Verify_Link (Target : access Shared_Element) return Boolean;


      procedure Initialize (Target : in out Shared_Element;
                            Value  : in     Element);
      procedure Initialize (Target : access Shared_Element;
                            Value  : in     Element);
      --  Note: Initialize is only safe to use when there are no
      --        concurrent updates.

   private

      ...  --  Implementation specific

   end Load_Linked_Store_Conditional;

   procedure Print_Statistics;

end NBAda.Large_Primitives;
\end{lstlisting}
\end{AdaCode}

\paragraph{Application constraints:}
\begin{itemize}
\item All objects of type \code{Shared\_Element} must be initialized
  with the operation \code{Initialize} before use.
\item Any task that calls an operation in
  \code{NBAda.Large\_Primitives} must have registered an
  identity by calling the operation \code{Register} of the
  appropriate instance of \code{NBAda.Process\_Identification}.
\end{itemize}


%%%%%%%%%%%%%%%%%%%%%%%%%%%%%%%%%%%%%%%%%%%%%%%%%%%%%%%%%%%%%%%%%%%%%%
\section{Containers}

\nbada\ includes a number of lock-free concurrent container data structures.

%%%%%%%%%%%%%%%%%%%%%%%%%%%%%%%%%%%%%%%%%%%%%%%%%%%%%%%%%%%%%%%%%
\subsection{Stacks}\label{chpt:LFSTACKS}

%%%%%%%%%%%%%%%%%%%%%%%%%%%%%%%%%%%%%%%%%%%%%%%%%%%%%%%%%%%%
\subsubsection*{The package NBAda.Lock\_Free\_Stacks}

The package \code{NBAda.Lock\_Free\_Stacks} implements a lock-free
unbounded stack data structure using an old well-known algorithm
\cite{IBM:1983,Michael:2004:SMR}.
It can use either the
\code{NBAda.Hazard\_Pointers} (\cmd{LF\_STACKS\_HPMR}) or
\code{NBAda.Epoch\_Based\_Memory\_Reclamation} (\cmd{LF\_STACKS\_EBMR})
memory reclamation algorithms.

\begin{AdaCode}
\begin{lstlisting}
generic
   type Element_Type is private;

   with package Process_Ids is
     new NBAda.Process_Identification (<>);
   --  Process identification.
package NBAda.Lock_Free_Stack is

   type Stack_Type is limited private;

   Stack_Empty : exception;

   procedure Push (On      : in out Stack_Type;
                   Element : in     Element_Type);
   procedure Pop  (From    : in out Stack_Type;
                   Element :    out Element_Type);
   function  Pop  (From    : access Stack_Type)
                  return Element_Type;

   function  Top  (From    : access Stack_Type)
                  return Element_Type;

private

      ...  --  Implementation specific

end NBAda.Lock_Free_Stack;
\end{lstlisting}
\end{AdaCode}

\paragraph{Application constraints:}
\begin{itemize}
\item Any task that calls an operation in \code{NBAda.Lock\_Free\_Stack}
  must have registered an identity by calling the operation
  \code{Register} of the appropriate instance of
  \code{NBAda.Process\_Identification}.
\end{itemize}


%%%%%%%%%%%%%%%%%%%%%%%%%%%%%%%%%%%%%%%%%%%%%%%%%%%%%%%%%%%%%%%%%
\subsection{Queues}\label{chpt:LFQUEUES}

%%%%%%%%%%%%%%%%%%%%%%%%%%%%%%%%%%%%%%%%%%%%%%%%%%%%%%%%%%%%
\subsubsection*{The package NBAda.Lock\_Free\_Bounded\_Queues}

\nbada\ contains a lock-free bounded size queue data structure based on
the algorithm by Tsigas and Zhang~\cite{TsigasZhang2001:SPAA}.

The include flag for \cmd{nbada\_config} is
\cmd{LF\_QUEUES\_BOUNDED}.

\begin{AdaCode}
\begin{lstlisting}
generic
   type Element_Type is private;
   --  The Element_Type must be atomic and Element_Type'Object_Size must be
   --  equal to System.Word_Size.
   Null_0 : Element_Type;
   Null_1 : Element_Type;
   --  NOTE: These two values MUST be different and MUST NOT appear as
   --        data values in the queue.
package NBAda.Lock_Free_Bounded_Queues is

   type Queue_Size is mod 2**32;

   type Lock_Free_Queue (Max_Size : Queue_Size) is limited private;

   procedure Enqueue (Queue   : in out Lock_Free_Queue;
                      Element : in     Element_Type);

   procedure Dequeue (Queue   : in out Lock_Free_Queue;
                      Element :    out Element_Type);

   function  Dequeue (Queue : access Lock_Free_Queue) return Element_Type;

   function Is_Empty (Queue : access Lock_Free_Queue) return Boolean;

   procedure Make_Empty (Queue : in out Lock_Free_Queue);
   --  NOTE: Make_Empty SHOULD NOT be used when concurrent access is possible.

   Queue_Full  : exception;
   Queue_Empty : exception;

private

      ...  --  Implementation specific

end NBAda.Lock_Free_Bounded_Queues;
\end{lstlisting}
\end{AdaCode}

\paragraph{Application constraints:}
\begin{itemize}
\item The type \code{Element\_Type} must be atomic.
\item \code{Element\_Type'Object\_Size} must be equal to
  \code{System.Word\_Size}.
\item The values passed as the two generic formal parameters
  \code{Null\_0} and \code{Null\_1} MUST be different and MUST NOT
  appear as data values in the queue.
\item The operation \code{Make\_Empty} SHOULD NOT be used when concurrent
  access to the queue object is possible.
\end{itemize}

%%%%%%%%%%%%%%%%%%%%%%%%%%%%%%%%%%%%%%%%%%%%%%%%%%%%%%%%%%%%
\subsubsection*{The package NBAda.Lock\_Free\_Queues}

\nbada\ contains two lock-free implementations of dynamic queues, one based on
the algorithm by Michael~\cite{Michael:1996:PODC} and one on the algorithm by
Hoffman et al.~\cite{HoffmanShSh:2007:TBQ}.

The include flag for \cmd{nbada\_config} is for Michael's queue algorithm
\cmd{LF\_QUEUES\_HPMR} or \cmd{LF\_QUEUES\_EBMR} and for Hoffman et al.'s queue
algorithm \cmd{LF\_QUEUES\_LFMR} or \cmd{LF\_QUEUES\_LFRC}.

\begin{AdaCode}
\begin{lstlisting}
generic
   type Element_Type is private;

   with package Process_Ids is
     new Process_Identification (<>);
   --  Process identification.
package NBAda.Lock_Free_Queues is

   type Queue_Type is limited private;

   Queue_Empty : exception;

   procedure Init    (Queue : in out Queue_Type);
   function  Dequeue (From : access Queue_Type) return Element_Type;
   procedure Enqueue (On      : in out Queue_Type;
                      Element : in     Element_Type);

private

      ...  --  Implementation specific

end NBAda.Lock_Free_Queues;
\end{lstlisting}
\end{AdaCode}

\paragraph{Application constraints:}
\begin{itemize}
\item Any task that calls an operation in \code{NBAda.Lock\_Free\_Queues}
  must have registered an identity by calling the operation
  \code{Register} of the appropriate instance of
  \code{NBAda.Process\_Identification}.
\item The operation \code{Init} SHOULD NOT be used when concurrent
  access to the queue object is possible.
\end{itemize}

%%%%%%%%%%%%%%%%%%%%%%%%%%%%%%%%%%%%%%%%%%%%%%%%%%%%%%%%%%%%%%%%%
\subsection{Deques}\label{chpt:LFDEQUES}

%%%%%%%%%%%%%%%%%%%%%%%%%%%%%%%%%%%%%%%%%%%%%%%%%%%%%%%%%%%%
\subsubsection*{The package NBAda.Lock\_Free\_Deques}

The package \code{NBAda.Lock\_Free\_Deques} implements a lock-free
unbounded double ended queue data structure based on the algorithm by
Sundell and Tsigas~\cite{SunT04c}.

The include flag for \cmd{nbada\_config} is 
\cmd{LF\_DEQUES\_LFMR} or \cmd{LF\_DEQUES\_LFRC}.

\begin{AdaCode}
\begin{lstlisting}
generic
   type Element_Type is private;

   with package Process_Ids is
     new Process_Identification (<>);
   --  Process identification.
package NBAda.Lock_Free_Deques is

   type Deque_Type is limited private;

   Deque_Empty : exception;

   procedure Init    (Deque : in out Deque_Type);

   function  Pop_Right  (Deque   : access Deque_Type) return Element_Type;
   procedure Push_Right (Deque   : in out Deque_Type;
                         Element : in     Element_Type);

   function  Pop_Left  (Deque   : access Deque_Type) return Element_Type;
   procedure Push_Left (Deque   : in out Deque_Type;
                        Element : in     Element_Type);

private

      ...  --  Implementation specific

end NBAda.Lock_Free_Deques;
\end{lstlisting}
\end{AdaCode}

\paragraph{Application constraints:}
\begin{itemize}
\item Any task that calls an operation in \code{NBAda.Lock\_Free\_Deques}
  must have registered an identity by calling the operation
  \code{Register} of the appropriate instance of
  \code{NBAda.Process\_Identification}.
\item The operation \code{Init} SHOULD NOT be used when concurrent
  access to the queue object is possible.
\end{itemize}

%%%%%%%%%%%%%%%%%%%%%%%%%%%%%%%%%%%%%%%%%%%%%%%%%%%%%%%%%%%%%%%%%
\subsection{Priority Queues}\label{chpt:LFPRIORITYQUEUES}

%%%%%%%%%%%%%%%%%%%%%%%%%%%%%%%%%%%%%%%%%%%%%%%%%%%%%%%%%%%%
\subsubsection*{The package NBAda.Lock\_Free\_Priority\_Queues}

\nbada\ contains a lock-free dynamic priority queue data structure
based on my (unpublished) modification of Michael's
list-based lock-free set algorithm~\cite{Michael:2002:HPDLF}.

The include flag for \cmd{nbada\_config} is
\cmd{LF\_PRIORITY\_QUEUES\_EBMR} or \cmd{LF\_PRIORITY\_QUEUES\_HPMR}.

\begin{AdaCode}
\begin{lstlisting}
generic

   type Element_Type is private;

   with function "<" (Left, Right : Element_Type) return Boolean is <>;
   --  Note: Element_Type must be totally ordered.

   with package Process_Ids is
     new Process_Identification (<>);
   --  Process identification.

package NBAda.Lock_Free_Priority_Queues is

   type Priority_Queue_Type is limited private;

   Queue_Empty     : exception;
   Already_Present : exception;

   procedure Initialize (Queue : in out Priority_Queue_Type);

   procedure Insert  (Into    : in out Priority_Queue_Type;
                      Element : in     Element_Type);

   procedure Delete_Min (From    : in out Priority_Queue_Type;
                         Element :    out Element_Type);
   function  Delete_Min (From : in Priority_Queue_Type)
                        return Element_Type;
   function  Delete_Min (From : access Priority_Queue_Type)
                        return Element_Type;

private

      ...  --  Implementation specific

end NBAda.Lock_Free_Priority_Queues;
\end{lstlisting}
\end{AdaCode}

\paragraph{Application constraints:}
\begin{itemize}
\item Any task that calls an operation in
  \code{NBAda.Lock\_Free\_Priority\_Queues} must have registered an
  identity by calling the operation \code{Register} of the appropriate
  instance of \code{NBAda.Process\_Identification}.
\item The function \code{"$<$"} on \code{Element\_Type} MUST define a total
  order.
\item The operation \code{Initialize} SHOULD NOT be used when concurrent
  access to the priority queue object is possible.
\end{itemize}


%%%%%%%%%%%%%%%%%%%%%%%%%%%%%%%%%%%%%%%%%%%%%%%%%%%%%%%%%%%%%%%%%
\subsection{Dictionaries and Sets}\label{chpt:LFMAPS}\label{chpt:LFSETS}

%%%%%%%%%%%%%%%%%%%%%%%%%%%%%%%%%%%%%%%%%%%%%%%%%%%%%%%%%%%%
\subsubsection*{The package NBAda.Lock\_Free\_Sets}

\nbada\ contains a lock-free dynamic set data structure based on the
list-based lock-free set algorithm by Michael~\cite{Michael:2002:HPDLF}.

The include flag for \cmd{nbada\_config} is
\cmd{LF\_SETS\_EBMR} or \cmd{LF\_SETS\_HPMR}.


\begin{AdaCode}
\begin{lstlisting}
generic

   type Value_Type is private;
   type Key_Type is private;

   with function "<" (Left, Right : Key_Type) return Boolean is <>;
   --  Note: Key_Type must be totally ordered.

   with package Process_Ids is
     new Process_Identification (<>);
   --  Process identification.

package NBAda.Lock_Free_Sets is
   type Set_Type is limited private;

   Not_Found       : exception;
   Already_Present : exception;

   procedure Init    (Set : in out Set_Type);

   procedure Insert  (Into  : in out Set_Type;
                      Key   : in     Key_Type;
                      Value : in     Value_Type);

   procedure Delete  (From : in out Set_Type;
                      Key  : in     Key_Type);

   function  Find    (In_Set : in Set_Type;
                      Key    : in Key_Type) return Value_Type;

private

      ...  --  Implementation specific

end NBAda.Lock_Free_Sets;
\end{lstlisting}
\end{AdaCode}

\paragraph{Application constraints:}
\begin{itemize}
\item Any task that calls an operation in
  \code{NBAda.Lock\_Free\_Sets} must have registered an
  identity by calling the operation \code{Register} of the appropriate
  instance of \code{NBAda.Process\_Identification}.
\item The function \code{"$<$"} on \code{Element\_Type} MUST define a total
  order.
\item The operation \code{Init} SHOULD NOT be used when concurrent
  access to the set object is possible.
\end{itemize}

%%%%%%%%%%%%%%%%%%%%%%%%%%%%%%%%%%%%%%%%%%%%%%%%%%%%%%%%%%%%
\subsubsection*{The package NBAda.Lock\_Free\_Dictionaries}

\nbada\ contains a lock-free dynamic dictionary data structure based on the
lock-free hash table and set algorithms by Michael~\cite{Michael:2002:HPDLF}.

The include flag for \cmd{nbada\_config} is
\cmd{LF\_DICTIONARIES\_EBMR} or \cmd{LF\_DICTIONARIES\_HPMR}.


\begin{AdaCode}
\begin{lstlisting}
generic

   type Value_Type is private;
   type Key_Type is private;

   with function Hash (Key        : Key_Type;
                       Table_Size : Positive) return Natural;

   with function "<" (Left, Right : Key_Type) return Boolean is <>;
   --  Note: Key_Type must be totally ordered.

   with package Process_Ids is
     new NBAda.Process_Identification (<>);
   --  Process identification.

package NBAda.Lock_Free_Dictionaries is

   type Dictionary_Type (No_Buckets : Natural) is limited private;

   Not_Found       : exception;
   Already_Present : exception;

   procedure Init    (Dictionary : in out Dictionary_Type);

   procedure Insert  (Into  : in out Dictionary_Type;
                      Key   : in     Key_Type;
                      Value : in     Value_Type);

   procedure Delete  (From  : in out Dictionary_Type;
                      Key   : in     Key_Type);

   function  Lookup  (From  : in Dictionary_Type;
                      Key   : in Key_Type)
                     return Value_Type;

private

      ...  --  Implementation specific

end NBAda.Lock_Free_Dictionaries;
\end{lstlisting}
\end{AdaCode}

\paragraph{Application constraints:}
\begin{itemize}
\item Any task that calls an operation in
  \code{NBAda.Lock\_Free\_Dictionaries} must have registered an
  identity by calling the operation \code{Register} of the appropriate
  instance of \code{NBAda.Process\_Identification}.
\item The function \code{"$<$"} on \code{Element\_Type} MUST define a total
  order.
\item The function \code{Hash} MUST return a value in the
  range 0 .. \code{Table\_Size} for every value of \code{Key\_Type}.
\item The operation \code{Init} SHOULD NOT be used when concurrent
  access to the set object is possible.
\end{itemize}

%%%%%%%%%%%%%%%%%%%%%%%%%%%%%%%%%%%%%%%%%%%%%%%%%%%%%%%%%%%%%%%%%%%%%%%%%%%
\chapter{Support Packages}\label{chpt:Support-pkgs}


%%%%%%%%%%%%%%%%%%%%%%%%%%%%%%%%%%%%%%%%%%%%%%%%%%%%%%%%%%%%%%%%%%%%%%
\section{Memory Reclamation Algorithms}\label{chpt:Memory-reclamation}

In a concurrent program it is often not obvious when it is safe to
free a dynamically allocated block of memory (consider e.g. the case
when another thread holds a local pointer to the object). In the
absence of a concurrency safe (and lock-free) general garbage
collector there are efficient lock-free memory reclamation algorithms
that can solve this problem, provided that the application or data
structure use them to manage dynamically allocated nodes and the
references to them.

The memory reclamation algorithms distinguish the managed nodes into
\emph{live} nodes that are part of the logical state of the user data
structure and \emph{logically deleted} nodes that are not part of the
logical state of the user data structure. In some memory reclamation
algorithms the user data structure is expected to notify the memory
reclamation algorithm when a node changes state to logically deleted,
in others in can be deduced from reachability. The memory reclamation
algorithm will delay the actual reclamation of a logically deleted
until there cannot be any potentially accesses to the node form any
thread (using the memory reclamation API).

There are two different levels of service or``protection'' offered by
memory reclamation algorithms, I define them as follows:
\begin{itemize}
\item {\bf Reclamation safe private references}. The
  memory reclamation algorithm only safeguards nodes referenced by
  private (task local) references, i.e. does not safeguard shared
  references. The application needs to take care that the shared
  references it uses cannot reference logically deleted nodes.  E.g.
  applications can usually only follow (dereference) references in
  nodes it \emph{knows} are alive.

\item {\bf Reclamation safe private and shared references}.  The
  memory reclamation algorithm safeguards all private and shared
  references. The application can safely dereference any shared
  reference.
\end{itemize}
See~\cite{GiPaSuTs:2008:LFMR} for a more thorough treatment of
lock-free memory reclamation algorithms and their properties.

\nbada\ includes implementations of several memory reclamation
algorithms of both service levels.

 
%%%%%%%%%%%%%%%%%%%%%%%%%%%%%%%%%%%%%%%%%%%%%%%%%%%%%%%%%%%%%%%%%
\subsection{Reclamation safe private references}

\nbada\ includes implementations of the Hazard Pointers lock-free
memory reclamation algorithm by
Michael~\cite{Michael:2002:SMR,Michael:2004:SMR}
(\code{NBAda.Hazard\_Pointers}) and the epoch based
concurrent memory reclamation algorithm described
in~\cite{Fraser:2004:PLF,Hart:2005:CPMR}
(\code{NBAda.Epoch\_Based\_Memory\_Reclamation}).

The intention is that the two packages should be API compatible.

%%%%%%%%%%%%%%%%%%%%%%%%%%%%%%%%%%%%%%%%%%%%%%%%%%%%%%%%%%%%
\subsubsection*{The package NBAda.Hazard\_Pointers and\\
  the package NBAda.Epoch\_Based\_Memory\_Reclamation}

\paragraph{Application constraints:}
\begin{itemize}
\item Any task that calls an memory reclamation operation must have
  registered an identity by calling the operation \code{Register} of
  the appropriate instance of \code{NBAda.Process\_Identification}.
\remove{
\item Dereference of shared reference value, a.k.a link, contained in
  a dereferenced \emph{statically linked} shared block is not
  guaranteed to yield a valid reference value, the obtained value can
  point to a block that has already been reclaimed.
}
\end{itemize}

\begin{AdaCode}
\begin{lstlisting}
generic
   Max_Number_Of_Dereferences : Natural;
   --  Maximum number of simultaneously dereferenced links per thread.
   with package Process_Ids is
     new Process_Identification (<>);
   --  Process identification.

   Integrity_Checking : Boolean := False;
   --  Enable strong integrity checking.
   Verbose_Debug : Boolean := False;
   --  Enable verbose debug output.
package NBAda.Hazard_Pointers is

   type Managed_Node_Base is abstract tagged limited private;
   --  Inherit from this base type to create your own managed types.

   procedure Free (Object : access Managed_Node_Base) is abstract;

   generic
      type Managed_Node is new Managed_Node_Base with private;
   package Operations is

      type Shared_Reference is limited private;
      --  Note: All shared variables of type Shared_Reference MUST be
      --        declared atomic by 'pragma Atomic (Variable_Name);' .

      type Node_Access is access all Managed_Node;
      --  Note: There SHOULD NOT be any shared variables of type
      --        Node_Access.

      function  Dereference (Shared : access Shared_Reference)
                            return Node_Access;
      --  Note:

      procedure Release     (Local  : in Node_Access);
      --  Note: Each dereferenced shared pointer MUST be released
      --        eventually.

      procedure Delete      (Local  : in Node_Access);
      --  Note: Delete may only be called when the caller can
      --        guarantee that there are NO and WILL NOT BE any more shared
      --        references to the node. The memory management scheme makes
      --        sure the node is not freed until all local references have
      --        been released.

      function  Boolean_Compare_And_Swap (Shared    : access Shared_Reference;
                                          Old_Value : in     Node_Access;
                                          New_Value : in     Node_Access)
                                         return Boolean;

      procedure Value_Compare_And_Swap   (Shared    : access Shared_Reference;
                                          Old_Value : in     Node_Access;
                                          New_Value : in out Node_Access);

      procedure Void_Compare_And_Swap    (Shared    : access Shared_Reference;
                                          Old_Value : in     Node_Access;
                                          New_Value : in     Node_Access);


      procedure Initialize (Shared    : access Shared_Reference;
                            New_Value : in     Node_Access);
      --  Note: Initialize is only safe to use when there are no
      --        concurrent updates.

   private

      type Shared_Reference is new Node_Access;
      --  Note: All shared variables of type Shared_Reference MUST be
      --        declared atomic by 'pragma Atomic (Variable_Name);' .

   end Operations;


   type Shared_Reference_Base is limited private;
   --  For type separation between shared references to different
   --  managed types derive your own shared reference types from
   --  Shared_Reference_Base and instantiate the memory management
   --  operation package below for each of them.

   generic

      type Managed_Node is
        new Managed_Node_Base with private;

      type Shared_Reference is new Shared_Reference_Base;
      --  All shared variables of type Shared_Reference MUST be declared
      --  atomic by 'pragma Atomic (Variable_Name);' .

   package Reference_Operations is

      type Node_Access is access all Managed_Node;
      --  Note: There SHOULD NOT be any shared variables of type
      --        Node_Access.

      type Private_Reference is private;
      --  Note: There SHOULD NOT be any shared variables of type
      --        Private_Reference.
      Null_Reference : constant Private_Reference;
      --  Note: A marked null reference is not equal to Null_Reference.

      function  Dereference (Link : access Shared_Reference)
                            return Private_Reference;

      procedure Release (Node : in Private_Reference);

      function  "+"     (Node : in Private_Reference)
                        return Node_Access;
      function  Deref   (Node : in Private_Reference)
                        return Node_Access;

      function  Boolean_Compare_And_Swap (Link      : access Shared_Reference;
                                          Old_Value : in Private_Reference;
                                          New_Value : in Private_Reference)
                                         return Boolean;

      procedure Void_Compare_And_Swap    (Link      : access Shared_Reference;
                                          Old_Value : in Private_Reference;
                                          New_Value : in Private_Reference);

      procedure Delete  (Node : in Private_Reference);

      procedure Store   (Link : access Shared_Reference;
                         Node : in Private_Reference);
      --  Note: Store is only safe to use when there cannot be any
      --        concurrent updates to Link.

      generic
         type User_Node_Access is access Managed_Node;
         --  Select an appropriate (preferably non-blocking) storage
         --  pool by the "for User_Node_Access'Storage_Pool use ..."
         --  construct.
         --  Note: The nodes allocated in this way must have an
         --        implementation of Free that use the same storage pool.
      function Create return Private_Reference;
      --  Creates a new User_Node and returns a safe reference to it.

      procedure Mark      (Node : in out Private_Reference);
      function  Mark      (Node : in     Private_Reference)
                          return Private_Reference;
      procedure Unmark    (Node : in out Private_Reference);
      function  Unmark    (Node : in     Private_Reference)
                          return Private_Reference;
      function  Is_Marked (Node : in     Private_Reference)
                          return Boolean;

      function  Is_Marked (Node : in     Shared_Reference)
                          return Boolean;

      function "=" (Link : in     Shared_Reference;
                    Ref  : in     Private_Reference) return Boolean;
      function "=" (Ref  : in     Private_Reference;
                    Link : in     Shared_Reference) return Boolean;

   private

      ...  --  Implementation details.

   end Reference_Operations;

   procedure Print_Statistics;

private

   ...  --  Implementation details.

end NBAda.Hazard_Pointers;
\end{lstlisting}
\end{AdaCode}

%%%%%%%%%%%%%%%%%%%%%%%%%%%%%%%%%%%%%%%%%%%%%%%%%%%%%%%%%%%%%%%%%
\subsection{Reclamation safe private and shared references}

\nbada\ contains implementations of two memory reclamation algorithms
that safeguards all private and shared references. The two algorithms
are the lock-free reference counting algorithm SLFRC by Herlihy et
al.~\cite{Herlihy:2002:ROP,HerlihyLuMaMo:2002:DSLF,HerlihyLuMaMo:2005:NBMM}
(\code{NBAda.Lock\_Free\_Reference\_Counting}) and the lock-free
reclamation algorithm Beware \& Cleanup by Gidenstam et
al.~\cite{GiPaSuTs:2005:LFGC}
(\code{NBAda.Lock\_Free\_Memory\_Reclamation}).

%%%%%%%%%%%%%%%%%%%%%%%%%%%%%%%%%%%%%%%%%%%%%%%%%%%%%%%%%%%%
\subsubsection*{The package NBAda.Lock\_Free\_Reference\_Counting and\\
  the package NBAda.Lock\_Free\_Memory\_Reclamation}

\paragraph{Application constraints:}
\begin{itemize}
\item Any task that calls an memory reclamation operation must have
  registered an identity by calling the operation \code{Register} of
  the appropriate instance of \code{NBAda.Process\_Identification}.
\end{itemize}

\begin{AdaCode}
\begin{lstlisting}
generic

   Max_Number_Of_Dereferences : Natural;
   --  Maximum number of simultaneously dereferenced links per thread.

   Max_Number_Of_Links_Per_Node : Natural;
   --  Maximum number of links in a shared node.

   with package Process_Ids is
     new NBAda.Process_Identification (<>);
   --  Process identification.

   Max_Delete_List_Size         : Natural :=
     Process_Ids.Max_Number_Of_Processes ** 2 *
       (Max_Number_Of_Dereferences + Max_Number_Of_Links_Per_Node +
        Max_Number_Of_Links_Per_Node + 1);

   Clean_Up_Threshold           : Natural := Max_Delete_List_Size;
   --  The threshold on the delete list size for Clean_Up to be done.

   Scan_Threshold               : Natural := Clean_Up_Threshold;
   --  The threshold on the delete list size for Scan to be done.

   Collect_Statistics           : Boolean := True;
   --  Enable some statics gathering.

package NBAda.Lock_Free_Memory_Reclamation is

   type Managed_Node_Base is abstract tagged limited private;
   --  Inherit from this base type to create your own managed types.

   procedure Dispose  (Node       : access Managed_Node_Base;
                       Concurrent : in     Boolean) is abstract;
   --  Dispose should set all shared references inside the node to null.

   procedure Clean_Up (Node : access Managed_Node_Base) is abstract;
   --  Clean_Up should make sure that none of the shared references
   --  inside the node points to a node that was deleted at the point
   --  in time when Clean_Up was called.

   function Is_Deleted (Node : access Managed_Node_Base)
                       return Boolean;
   --  Returns true if Delete (see below) has been called on the node.

   procedure Free (Object : access Managed_Node_Base) is abstract;
   --  Note: Due to some peculiarities of the Ada storage pool
   --        management managed nodes need to have a dispatching primitive
   --        operation that calls the instance of Unchecked_Deallocation
   --        appropriate for the specific node type at hand. Without
   --        this the wrong instance of Unchecked_Deallocation might get
   --        called - often with disastrous consequences as it tries return
   --        the memory to the wrong storage pool.

   type Shared_Reference_Base is limited private;
   --  For type separation between shared references to different
   --  managed types derive your own shared reference types from
   --  Shared_Reference_Base and instantiate the memory management
   --  operation package below for each of them.

   type Shared_Reference_Base_Access is access all Shared_Reference_Base;
   type Reference_Set is array (Integer range <>) of
     Shared_Reference_Base_Access;
   --  These two types are defined for compatibility with the
   --  Lock_Free_Reference_Counting package.

   generic

      type Managed_Node is
        new Managed_Node_Base with private;

      type Shared_Reference is new Shared_Reference_Base;
      --  All shared variables of type Shared_Reference MUST be declared
      --  atomic by 'pragma Atomic (Variable_Name);' .

   package Operations is

      type Node_Access is access all Managed_Node;
      --  Note: There SHOULD NOT be any shared variables of type
      --        Node_Access.

      type Private_Reference is private;
      --  Note: There SHOULD NOT be any shared variables of type
      --        Private_Reference.
      Null_Reference : constant Private_Reference;
      function Image (R : Private_Reference) return String;

      function  Dereference (Link : access Shared_Reference)
                            return Private_Reference;

      procedure Release (Node : in Private_Reference);

      function  "+"     (Node : in Private_Reference)
                        return Node_Access;
      function  Deref   (Node : in Private_Reference)
                        return Node_Access;

      function  Copy (Node : in Private_Reference) return Private_Reference;
      --  Creates a new Private Reference to Node. Both will need to be
      --  released.

      function  Compare_And_Swap (Link      : access Shared_Reference;
                                  Old_Value : in Private_Reference;
                                  New_Value : in Private_Reference)
                                 return Boolean;

      procedure Compare_And_Swap (Link      : access Shared_Reference;
                                  Old_Value : in     Private_Reference;
                                  New_Value : in     Private_Reference);

      procedure Delete  (Node : in Private_Reference);


      procedure Store   (Link : access Shared_Reference;
                         Node : in Private_Reference);

      generic
         type User_Node_Access is access Managed_Node;
         --  Select an appropriate (preferably non-blocking) storage
         --  pool by the "for User_Node_Access'Storage_Pool use ..."
         --  construct.
         --  Note: The nodes allocated in this way must have an
         --        implementation of Free that use the same storage pool.
      function Create return Private_Reference;
      --  Creates a new User_Node and returns a safe reference to it.

      --  Private (and shared) references can be tagged with a mark.
      --  NOTE: A marked Null_Reference is not equal (=) to an unmarked.
      procedure Mark      (Node : in out Private_Reference);

      function  Mark      (Node : in     Private_Reference)
                          return Private_Reference;
      procedure Unmark    (Node : in out Private_Reference);
      function  Unmark    (Node : in     Private_Reference)
                          return Private_Reference;
      function  Is_Marked (Node : in     Private_Reference)
                          return Boolean;

      function  Is_Marked (Node : in     Shared_Reference)
                          return Boolean;

      function "=" (Left  : in     Private_Reference;
                    Right : in     Private_Reference) return Boolean;
      function "=" (Link : in     Shared_Reference;
                    Ref  : in     Private_Reference) return Boolean;
      function "=" (Ref  : in     Private_Reference;
                    Link : in     Shared_Reference) return Boolean;
      --  It is possible to compare a reference to the current value of a link.

      ------------------------------------------------------------------------
      --  Unsafe operations.
      --  These SHOULD only be use when the user algorithm guarantees
      --  the absence of ABA-problems.
      --  In such algorithms the use of these operations in some particular
      --  situations could allow some performance improving optimizations.
      ------------------------------------------------------------------------

      type Unsafe_Reference_Value is private;
      --  Note: An Unsafe_Reference_Value does not keep a claim to any
      --        node and can therefore only be used where ABA safety is
      --        ensured by other means. It cannot be dereferenced.

      function  Unsafe_Read (Link : access Shared_Reference)
                            return Unsafe_Reference_Value;

      function  Compare_And_Swap (Link      : access Shared_Reference;
                                  Old_Value : in Unsafe_Reference_Value;
                                  New_Value : in Private_Reference)
                                 return Boolean;
      function  Compare_And_Swap (Link      : access Shared_Reference;
                                  Old_Value : in Unsafe_Reference_Value;
                                  New_Value : in Unsafe_Reference_Value)
                                 return Boolean;
      procedure Compare_And_Swap (Link      : access Shared_Reference;
                                  Old_Value : in     Unsafe_Reference_Value;
                                  New_Value : in     Private_Reference);
      procedure Compare_And_Swap (Link      : access Shared_Reference;
                                  Old_Value : in     Unsafe_Reference_Value;
                                  New_Value : in     Unsafe_Reference_Value);

      function  Is_Marked (Node : in     Unsafe_Reference_Value)
                          return Boolean;

      function  Mark      (Node : in     Unsafe_Reference_Value)
                          return Unsafe_Reference_Value;

      function "=" (Val : in     Unsafe_Reference_Value;
                    Ref : in     Private_Reference) return Boolean;
      function "=" (Ref : in     Private_Reference;
                    Val : in     Unsafe_Reference_Value) return Boolean;

      function "=" (Link : in     Shared_Reference;
                    Ref  : in     Unsafe_Reference_Value) return Boolean;
      function "=" (Ref  : in     Unsafe_Reference_Value;
                    Link : in     Shared_Reference) return Boolean;

   private

      ...  --  Implementation details.

   end Operations;

   procedure Print_Statistics;

private

   ...  --  Implementation details.

end NBAda.Lock_Free_Memory_Reclamation;
\end{lstlisting}
\end{AdaCode}


%%%%%%%%%%%%%%%%%%%%%%%%%%%%%%%%%%%%%%%%%%%%%%%%%%%%%%%%%%%%%%%%%%%%%%
\section{Memory Allocation Pools}\label{chpt:Memory-allocation}

%\cite{Dice:2002:MLFM,Michael:2004:SLFDMA,GiPaTs:2005:AMLF,SchneiderAnNi:2006:SLCMMA}

%%%%%%%%%%%%%%%%%%%%%%%%%%%%%%%%%%%%%%%%%%%%%%%%%%%%%%%%%%%%
\subsubsection*{The package NBAda.Lock\_Free\_Fixed\_Size\_Storage\_Pools}

\nbada\ contains a generic fixed size lock-free storage pool based on the 
lock-free free-list algorithm in~\cite{IBM:1983}.

\paragraph{Application constraints:}
\begin{itemize}
\item A pool instance MUST NOT be used for object that have storage
  size larger than \code{Block\_Size}.
\end{itemize}

\begin{AdaCode}
\begin{lstlisting}
package NBAda.Lock_Free_Fixed_Size_Storage_Pools is

   type Block_Count is range 0 .. 2**16 - 1;

   type Lock_Free_Storage_Pool
     (Pool_Size  : Block_Count;
      Block_Size : System.Storage_Elements.Storage_Count) is
     new System.Storage_Pools.Root_Storage_Pool with private;

   procedure Allocate
     (Pool                     : in out Lock_Free_Storage_Pool;
      Storage_Address          :    out System.Address;
      Size_In_Storage_Elements : in     System.Storage_Elements.Storage_Count;
      Alignment                : in     System.Storage_Elements.Storage_Count);

   procedure Deallocate
     (Pool                     : in out Lock_Free_Storage_Pool;
      Storage_Address          : in     System.Address;
      Size_In_Storage_Elements : in     System.Storage_Elements.Storage_Count;
      Alignment                : in     System.Storage_Elements.Storage_Count);

   function Storage_Size (Pool : Lock_Free_Storage_Pool)
                         return System.Storage_Elements.Storage_Count;

   function Validate (Pool : Lock_Free_Storage_Pool)
                     return Block_Count;

   function Belongs_To (Pool            : Lock_Free_Storage_Pool;
                        Storage_Address : System.Address)
                       return Boolean;

   Storage_Exhausted    : exception;
   Implementation_Error : exception;

private

   ...  --  Implementation details.

end NBAda.Lock_Free_Fixed_Size_Storage_Pools;
\end{lstlisting}
\end{AdaCode}

%%%%%%%%%%%%%%%%%%%%%%%%%%%%%%%%%%%%%%%%%%%%%%%%%%%%%%%%%%%%
\subsubsection*{The package NBAda.Lock\_Free\_Growing\_Storage\_Pools}

The growing storage pool in \nbada\ automatically grows in size when
the memory demand warrants it. It never shirks, however.

\paragraph{Application constraints:}
\begin{itemize}
\item A pool instance MUST NOT be used for object that have storage
  size larger than \code{Block\_Size}.
\end{itemize}

\begin{AdaCode}
\begin{lstlisting}
package NBAda.Lock_Free_Growing_Storage_Pools is

   type Lock_Free_Storage_Pool
     (Block_Size : System.Storage_Elements.Storage_Count) is
     new System.Storage_Pools.Root_Storage_Pool with private;

   procedure Allocate
     (Pool                     : in out Lock_Free_Storage_Pool;
      Storage_Address          :    out System.Address;
      Size_In_Storage_Elements : in     System.Storage_Elements.Storage_Count;
      Alignment                : in     System.Storage_Elements.Storage_Count);

   procedure Deallocate
     (Pool                     : in out Lock_Free_Storage_Pool;
      Storage_Address          : in     System.Address;
      Size_In_Storage_Elements : in     System.Storage_Elements.Storage_Count;
      Alignment                : in     System.Storage_Elements.Storage_Count);

   function Storage_Size (Pool : Lock_Free_Storage_Pool)
                         return System.Storage_Elements.Storage_Count;

   function Validate (Pool : Lock_Free_Storage_Pool)
                     return Natural;

   Storage_Exhausted : exception;
   Implementation_Error : exception;

private

   ...  --  Implementation details.

end NBAda.Lock_Free_Growing_Storage_Pools;
\end{lstlisting}
\end{AdaCode}



%%%%%%%%%%%%%%%%%%%%%%%%%%%%%%%%%%%%%%%%%%%%%%%%%%%%%%%%%%%%%%%%%%%%%%
\section{Hardware Abstraction Interface}\label{chpt:Primitives}

%%%%%%%%%%%%%%%%%%%%%%%%%%%%%%%%%%%%%%%%%%%%%%%%%%%%%%%%%%%%
\subsubsection*{The package NBAda.Primitives}

\begin{AdaCode}
\begin{lstlisting}
package NBAda.Primitives is

   Not_Implemented : exception;

   procedure Membar;

   type Standard_Unsigned is mod 2**System.Word_Size;
   pragma Atomic (Standard_Unsigned);

   generic
      --  Element'Object_Size MUST be System.Word_Size.
      type Element is private;
   function Standard_Atomic_Read (Target : access Element) return Element;

   generic
      --  Element'Object_Size MUST be System.Word_Size.
      type Element is private;
   procedure Standard_Atomic_Write (Target : access Element;
                                    Value  : in     Element);

   generic
      --  Element'Object_Size MUST be System.Word_Size.
      type Element is private;
   procedure Standard_Compare_And_Swap (Target    : access Element;
                                        Old_Value : in     Element;
                                        New_Value : in out Element);

   generic
      --  Element'Object_Size MUST be System.Word_Size.
      type Element is private;
   function Standard_Boolean_Compare_And_Swap (Target    : access Element;
                                               Old_Value : in     Element;
                                               New_Value : in     Element)
                                              return Boolean;

   generic
      --  Element'Object_Size MUST be System.Word_Size.
      type Element is private;
   procedure Standard_Void_Compare_And_Swap (Target    : access Element;
                                             Old_Value : in     Element;
                                             New_Value : in     Element);

   procedure Fetch_And_Add (Target    : access Standard_Unsigned;
                            Increment : in     Standard_Unsigned);

   function  Fetch_And_Add (Target    : access Standard_Unsigned;
                            Increment : in     Standard_Unsigned)
                           return Standard_Unsigned;


   type Unsigned_32 is mod 2**32;
   pragma Atomic (Unsigned_32);

   generic
      --  Element'Object_Size MUST be 32.
      type Element is private;
   function Atomic_Read_32 (Target : access Element) return Element;

   generic
      --  Element'Object_Size MUST be 32.
      type Element is private;
   procedure Atomic_Write_32 (Target : access Element;
                              Value  : in     Element);

   generic
      --  Element'Object_Size MUST be 32.
      type Element is private;
   procedure Compare_And_Swap_32 (Target    : access Element;
                                  Old_Value : in     Element;
                                  New_Value : in out Element);

   generic
      --  Element'Object_Size MUST be 32.
      type Element is private;
   function Boolean_Compare_And_Swap_32 (Target    : access Element;
                                         Old_Value : in     Element;
                                         New_Value : in     Element)
                                        return Boolean;

   generic
      --  Element'Object_Size MUST be 32.
      type Element is private;
   procedure Void_Compare_And_Swap_32 (Target    : access Element;
                                       Old_Value : in     Element;
                                       New_Value : in     Element);

   procedure Fetch_And_Add_32 (Target    : access Unsigned_32;
                               Increment : in     Unsigned_32);

   function  Fetch_And_Add_32 (Target    : access Unsigned_32;
                               Increment : in     Unsigned_32)
                              return Unsigned_32;

   type Unsigned_64 is mod 2**64;
   pragma Atomic (Unsigned_64);

   generic
      --  Element'Object_Size MUST be 64.
      type Element is private;
   function Atomic_Read_64 (Target : access Element) return Element;

   generic
      --  Element'Object_Size MUST be 64.
      type Element is private;
   procedure Atomic_Write_64 (Target : access Element;
                              Value  : in     Element);

   generic
      --  Element'Object_Size MUST be 64.
      type Element is private;
   procedure Compare_And_Swap_64 (Target    : access Element;
                                  Old_Value : in     Element;
                                  New_Value : in out Element);

   generic
      --  Element'Object_Size MUST be 64.
      type Element is private;
   function Boolean_Compare_And_Swap_64 (Target    : access Element;
                                         Old_Value : in     Element;
                                         New_Value : in     Element)
                                        return Boolean;

   generic
      --  Element'Object_Size MUST be 64.
      type Element is private;
   procedure Void_Compare_And_Swap_64 (Target    : access Element;
                                       Old_Value : in     Element;
                                       New_Value : in     Element);

   procedure Fetch_And_Add_64 (Target    : access Unsigned_64;
                               Increment : in     Unsigned_64);

   function  Fetch_And_Add_64 (Target    : access Unsigned_64;
                               Increment : in     Unsigned_64)
                              return Unsigned_64;

end NBAda.Primitives;
\end{lstlisting}
\end{AdaCode}

%%%%%%%%%%%%%%%%%%%%%%%%%%%%%%%%%%%%%%%%%%%%%%%%%%%%%%%%%%%%
\subsubsection*{The package NBAda.Process\_Identification}

\begin{AdaCode}
\begin{lstlisting}
generic
   Max_Number_Of_Processes : Natural;
package NBAda.Process_Identification is

   type Process_ID_Type is new Natural range 1 .. Max_Number_Of_Processes;

   --  Register a process ID for this task.
   procedure Register;

   --  Returns the process ID of the calling task.
   function Process_ID return Process_ID_Type;

end NBAda.Process_Identification;
\end{lstlisting}
\end{AdaCode}

%%%%%%%%%%%%%%%%%%%%%%%%%%%%%%%%%%%%%%%%%%%%%%%%%%%%%%%%%%%%%%%%%%%%%%%%%%%
%% References and index.
\cleardoublepage
\addcontentsline{toc}{chapter}{Bibliography}
\bibliographystyle{alpha}
\bibliography{references}

%%%%%%%%%%%%%%%%%%%%%%%%%%%%%%%%%%%%%%%%%%%%%%%%%%%%%%%%%%%%%%%%%%%%%%%%%%%
\appendix
\chapter{GNU General Public Licence}\label{chpt:License}

\begin{TextOutput}
\begin{verbatim}
		    GNU GENERAL PUBLIC LICENSE
		       Version 2, June 1991

 Copyright (C) 1989, 1991 Free Software Foundation, Inc.
  59 Temple Place, Suite 330, Boston, MA  02111-1307  USA
 Everyone is permitted to copy and distribute verbatim copies
 of this license document, but changing it is not allowed.

			    Preamble

  The licenses for most software are designed to take away your
freedom to share and change it.  By contrast, the GNU General Public
License is intended to guarantee your freedom to share and change free
software--to make sure the software is free for all its users.  This
General Public License applies to most of the Free Software
Foundation's software and to any other program whose authors commit to
using it.  (Some other Free Software Foundation software is covered by
the GNU Library General Public License instead.)  You can apply it to
your programs, too.

  When we speak of free software, we are referring to freedom, not
price.  Our General Public Licenses are designed to make sure that you
have the freedom to distribute copies of free software (and charge for
this service if you wish), that you receive source code or can get it
if you want it, that you can change the software or use pieces of it
in new free programs; and that you know you can do these things.

  To protect your rights, we need to make restrictions that forbid
anyone to deny you these rights or to ask you to surrender the rights.
These restrictions translate to certain responsibilities for you if you
distribute copies of the software, or if you modify it.

  For example, if you distribute copies of such a program, whether
gratis or for a fee, you must give the recipients all the rights that
you have.  You must make sure that they, too, receive or can get the
source code.  And you must show them these terms so they know their
rights.

  We protect your rights with two steps: (1) copyright the software, and
(2) offer you this license which gives you legal permission to copy,
distribute and/or modify the software.

  Also, for each author's protection and ours, we want to make certain
that everyone understands that there is no warranty for this free
software.  If the software is modified by someone else and passed on, we
want its recipients to know that what they have is not the original, so
that any problems introduced by others will not reflect on the original
authors' reputations.

  Finally, any free program is threatened constantly by software
patents.  We wish to avoid the danger that redistributors of a free
program will individually obtain patent licenses, in effect making the
program proprietary.  To prevent this, we have made it clear that any
patent must be licensed for everyone's free use or not licensed at all.

  The precise terms and conditions for copying, distribution and
modification follow.

		    GNU GENERAL PUBLIC LICENSE
   TERMS AND CONDITIONS FOR COPYING, DISTRIBUTION AND MODIFICATION

  0. This License applies to any program or other work which contains
a notice placed by the copyright holder saying it may be distributed
under the terms of this General Public License.  The "Program", below,
refers to any such program or work, and a "work based on the Program"
means either the Program or any derivative work under copyright law:
that is to say, a work containing the Program or a portion of it,
either verbatim or with modifications and/or translated into another
language.  (Hereinafter, translation is included without limitation in
the term "modification".)  Each licensee is addressed as "you".

Activities other than copying, distribution and modification are not
covered by this License; they are outside its scope.  The act of
running the Program is not restricted, and the output from the Program
is covered only if its contents constitute a work based on the
Program (independent of having been made by running the Program).
Whether that is true depends on what the Program does.

  1. You may copy and distribute verbatim copies of the Program's
source code as you receive it, in any medium, provided that you
conspicuously and appropriately publish on each copy an appropriate
copyright notice and disclaimer of warranty; keep intact all the
notices that refer to this License and to the absence of any warranty;
and give any other recipients of the Program a copy of this License
along with the Program.

You may charge a fee for the physical act of transferring a copy, and
you may at your option offer warranty protection in exchange for a fee.

  2. You may modify your copy or copies of the Program or any portion
of it, thus forming a work based on the Program, and copy and
distribute such modifications or work under the terms of Section 1
above, provided that you also meet all of these conditions:

    a) You must cause the modified files to carry prominent notices
    stating that you changed the files and the date of any change.

    b) You must cause any work that you distribute or publish, that in
    whole or in part contains or is derived from the Program or any
    part thereof, to be licensed as a whole at no charge to all third
    parties under the terms of this License.

    c) If the modified program normally reads commands interactively
    when run, you must cause it, when started running for such
    interactive use in the most ordinary way, to print or display an
    announcement including an appropriate copyright notice and a
    notice that there is no warranty (or else, saying that you provide
    a warranty) and that users may redistribute the program under
    these conditions, and telling the user how to view a copy of this
    License.  (Exception: if the Program itself is interactive but
    does not normally print such an announcement, your work based on
    the Program is not required to print an announcement.)

These requirements apply to the modified work as a whole.  If
identifiable sections of that work are not derived from the Program,
and can be reasonably considered independent and separate works in
themselves, then this License, and its terms, do not apply to those
sections when you distribute them as separate works.  But when you
distribute the same sections as part of a whole which is a work based
on the Program, the distribution of the whole must be on the terms of
this License, whose permissions for other licensees extend to the
entire whole, and thus to each and every part regardless of who wrote it.

Thus, it is not the intent of this section to claim rights or contest
your rights to work written entirely by you; rather, the intent is to
exercise the right to control the distribution of derivative or
collective works based on the Program.

In addition, mere aggregation of another work not based on the Program
with the Program (or with a work based on the Program) on a volume of
a storage or distribution medium does not bring the other work under
the scope of this License.

  3. You may copy and distribute the Program (or a work based on it,
under Section 2) in object code or executable form under the terms of
Sections 1 and 2 above provided that you also do one of the following:

    a) Accompany it with the complete corresponding machine-readable
    source code, which must be distributed under the terms of Sections
    1 and 2 above on a medium customarily used for software interchange; or,

    b) Accompany it with a written offer, valid for at least three
    years, to give any third party, for a charge no more than your
    cost of physically performing source distribution, a complete
    machine-readable copy of the corresponding source code, to be
    distributed under the terms of Sections 1 and 2 above on a medium
    customarily used for software interchange; or,

    c) Accompany it with the information you received as to the offer
    to distribute corresponding source code.  (This alternative is
    allowed only for noncommercial distribution and only if you
    received the program in object code or executable form with such
    an offer, in accord with Subsection b above.)

The source code for a work means the preferred form of the work for
making modifications to it.  For an executable work, complete source
code means all the source code for all modules it contains, plus any
associated interface definition files, plus the scripts used to
control compilation and installation of the executable.  However, as a
special exception, the source code distributed need not include
anything that is normally distributed (in either source or binary
form) with the major components (compiler, kernel, and so on) of the
operating system on which the executable runs, unless that component
itself accompanies the executable.

If distribution of executable or object code is made by offering
access to copy from a designated place, then offering equivalent
access to copy the source code from the same place counts as
distribution of the source code, even though third parties are not
compelled to copy the source along with the object code.

  4. You may not copy, modify, sublicense, or distribute the Program
except as expressly provided under this License.  Any attempt
otherwise to copy, modify, sublicense or distribute the Program is
void, and will automatically terminate your rights under this License.
However, parties who have received copies, or rights, from you under
this License will not have their licenses terminated so long as such
parties remain in full compliance.

  5. You are not required to accept this License, since you have not
signed it.  However, nothing else grants you permission to modify or
distribute the Program or its derivative works.  These actions are
prohibited by law if you do not accept this License.  Therefore, by
modifying or distributing the Program (or any work based on the
Program), you indicate your acceptance of this License to do so, and
all its terms and conditions for copying, distributing or modifying
the Program or works based on it.

  6. Each time you redistribute the Program (or any work based on the
Program), the recipient automatically receives a license from the
original licensor to copy, distribute or modify the Program subject to
these terms and conditions.  You may not impose any further
restrictions on the recipients' exercise of the rights granted herein.
You are not responsible for enforcing compliance by third parties to
this License.

  7. If, as a consequence of a court judgment or allegation of patent
infringement or for any other reason (not limited to patent issues),
conditions are imposed on you (whether by court order, agreement or
otherwise) that contradict the conditions of this License, they do not
excuse you from the conditions of this License.  If you cannot
distribute so as to satisfy simultaneously your obligations under this
License and any other pertinent obligations, then as a consequence you
may not distribute the Program at all.  For example, if a patent
license would not permit royalty-free redistribution of the Program by
all those who receive copies directly or indirectly through you, then
the only way you could satisfy both it and this License would be to
refrain entirely from distribution of the Program.

If any portion of this section is held invalid or unenforceable under
any particular circumstance, the balance of the section is intended to
apply and the section as a whole is intended to apply in other
circumstances.

It is not the purpose of this section to induce you to infringe any
patents or other property right claims or to contest validity of any
such claims; this section has the sole purpose of protecting the
integrity of the free software distribution system, which is
implemented by public license practices.  Many people have made
generous contributions to the wide range of software distributed
through that system in reliance on consistent application of that
system; it is up to the author/donor to decide if he or she is willing
to distribute software through any other system and a licensee cannot
impose that choice.

This section is intended to make thoroughly clear what is believed to
be a consequence of the rest of this License.

  8. If the distribution and/or use of the Program is restricted in
certain countries either by patents or by copyrighted interfaces, the
original copyright holder who places the Program under this License
may add an explicit geographical distribution limitation excluding
those countries, so that distribution is permitted only in or among
countries not thus excluded.  In such case, this License incorporates
the limitation as if written in the body of this License.

  9. The Free Software Foundation may publish revised and/or new versions
of the General Public License from time to time.  Such new versions will
be similar in spirit to the present version, but may differ in detail to
address new problems or concerns.

Each version is given a distinguishing version number.  If the Program
specifies a version number of this License which applies to it and "any
later version", you have the option of following the terms and conditions
either of that version or of any later version published by the Free
Software Foundation.  If the Program does not specify a version number of
this License, you may choose any version ever published by the Free Software
Foundation.

  10. If you wish to incorporate parts of the Program into other free
programs whose distribution conditions are different, write to the author
to ask for permission.  For software which is copyrighted by the Free
Software Foundation, write to the Free Software Foundation; we sometimes
make exceptions for this.  Our decision will be guided by the two goals
of preserving the free status of all derivatives of our free software and
of promoting the sharing and reuse of software generally.

			    NO WARRANTY

  11. BECAUSE THE PROGRAM IS LICENSED FREE OF CHARGE, THERE IS NO WARRANTY
FOR THE PROGRAM, TO THE EXTENT PERMITTED BY APPLICABLE LAW.  EXCEPT WHEN
OTHERWISE STATED IN WRITING THE COPYRIGHT HOLDERS AND/OR OTHER PARTIES
PROVIDE THE PROGRAM "AS IS" WITHOUT WARRANTY OF ANY KIND, EITHER EXPRESSED
OR IMPLIED, INCLUDING, BUT NOT LIMITED TO, THE IMPLIED WARRANTIES OF
MERCHANTABILITY AND FITNESS FOR A PARTICULAR PURPOSE.  THE ENTIRE RISK AS
TO THE QUALITY AND PERFORMANCE OF THE PROGRAM IS WITH YOU.  SHOULD THE
PROGRAM PROVE DEFECTIVE, YOU ASSUME THE COST OF ALL NECESSARY SERVICING,
REPAIR OR CORRECTION.

  12. IN NO EVENT UNLESS REQUIRED BY APPLICABLE LAW OR AGREED TO IN WRITING
WILL ANY COPYRIGHT HOLDER, OR ANY OTHER PARTY WHO MAY MODIFY AND/OR
REDISTRIBUTE THE PROGRAM AS PERMITTED ABOVE, BE LIABLE TO YOU FOR DAMAGES,
INCLUDING ANY GENERAL, SPECIAL, INCIDENTAL OR CONSEQUENTIAL DAMAGES ARISING
OUT OF THE USE OR INABILITY TO USE THE PROGRAM (INCLUDING BUT NOT LIMITED
TO LOSS OF DATA OR DATA BEING RENDERED INACCURATE OR LOSSES SUSTAINED BY
YOU OR THIRD PARTIES OR A FAILURE OF THE PROGRAM TO OPERATE WITH ANY OTHER
PROGRAMS), EVEN IF SUCH HOLDER OR OTHER PARTY HAS BEEN ADVISED OF THE
POSSIBILITY OF SUCH DAMAGES.

		     END OF TERMS AND CONDITIONS
\end{verbatim}
\end{TextOutput}

%%%%%%%%%%%%%%%%%%%%%%%%%%%%%%%%%%%%%%%%%%%%%%%%%%%%%%%%%%%%%%%%%%%%%%%%%%%
%\printindex


\end{document}